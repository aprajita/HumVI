%=================================================

% Notes on human-viewable (and robot-viewable) images,
% information content and so on. 

% These are thoughts in progress -- hopefully this text
% will be useful at some point, chopped up and rearranged
% into a coherent story!

\documentclass[letterpaper, 11pt]{article}

%=================================================

\usepackage{amsmath,amssymb}

\title{Viewing Informative Images}
\author{Phil Marshall\thanks{\texttt{dr.phil.marshall@gmail.edu}}}
\date{\today}

%%-------------------------------------------------

\begin{document}

\maketitle

\begin{abstract} 

In astronomy, and many other areas of science, discoveries can be made simply
by looking at images. While this might be thought of as data exploration, we
argue that it is actually a form of data modeling -- and that  by considering
a robotic viewer who computes and reports the Shannon information content of
images relative to its model for how they were generated, we arrive at some
hypotheses for how one might prepare images for a human image inspection
program that has high capability for discovery.

\end{abstract}


% =================================================

\section{Introduction}

Images contain information. In astronomy, new planets, stars and galaxies can
be discovered simply by looking at images taken of the sky through telescopes
-- a statement which can be generalised to many other fields of science. How
is this possible? For something new to be discovered in an image, that new
feature must, by definition, be something that has not been seen before:
something that is at odds with our expectations of the features that similar 
images contain, something that is surprising. Image viewing is a form of data
exploration, an important step before making quantitative measurements. But
both exploration and measurement involve data modeling, or inference: the
viewer necessarily interprets the image in the context of a model for that
image that they have in their heads. The surprisingness of a feature in an
image is relative to this model: discovery is relative to what is already
known. Note that in general discovery is a personal experience: each viewer
has their own model for image features against which new images are compared.
Only if the model is shared among (and agreed upon by) all viewers will a
discovery be global: if making such global discoveries is our goal as
scientists, then we must make this model as complete as possible, and publish
it. This is what astronomers do.

A mathematical description of discovery must therefore relate the data in hand
to the model in mind, and will provide a quantification of that new
information which constitutes the discovery.  In probability theory, the
(Shannon) information is defined as $I = -\log_2 P$, where $P$ is a
probability. $I$ is a measure of surprise: the lower an event's probability,
the more surprising it is, and the higher is its information content $I$. We
can  compute the  information content of a dataset $\boldsymbol{d}$ using this
definition, such that $I(\boldsymbol{d}|H)$ quantifies how surprising that
dataset is to its observer, who (naturally)  has a set of expectations $H$
about it.  Specifically, the information content of the dataset (in bits) is
given by
\begin{equation}
I(\boldsymbol{d}|H) = -\log_2 {\rm Pr}(\boldsymbol{d}|H).
\end{equation}  
This particular probability (also known as the evidence, or marginal
likelihood) is an integral over the (typically immense) parameter space of the
model, 
${\rm Pr}(\boldsymbol{d}|H) = 
\int {\rm Pr}(\boldsymbol{d}|\boldsymbol{\theta},H)
     {\rm Pr}(\boldsymbol{\theta}|H)$. In this expression 
we see two familiar quantities, the likelihood of the data, and the prior PDF
over the model parameters.  The negative sign indicates that information
content is highest when the marginal likelihood is {\it minimized}. That is,
datasets  that are fitted terribly by the model, even when considering all
possible  values of the parameters of the model, are highly informative. 
Further, if the data quality improves, then the goodness of fit to this
terrible model  will get {\it worse}, in that the significance of the
residuals will increase -- which means a decrease in the likelihood, and an
increase in the information content of the dataset. In this way,  information
content can be high because the data is good, or the model is bad -- or both.
 
What does this mean for image viewing?  Let us consider a robot whose purpose
is to view images and report new discoveries from them. We could design it
such that it computes and reports the information $I$ associated with each
feature in the image, based on an internal model for astronomical images that
we provide.  Clearly, if $I$ is to be a useful metric for the robot to use
when  evaluating a dataset, the model $H$ has to be as complete as possible:
it has to represent everything that we know about the features present in 
astronomical images, including the probability distributions for the
parameters that describe those features. If $H$ is incomplete in some way, the
robot will be surprised by features that we find familiar. Only when $H$ is
complete will the robot be efficient at discovery, and report features that
are new to all of us as well as to itself. Note that in science we seek both
discovery and understanding, which means maximizing information content with
respect to $\boldsymbol{d}$, but minimizing it with respect to $H$: for a
given dataset we want to find the model that agrees with it the most, but for
a given model we want the dataset that agrees with it the least. 

Our working supposition is that the robot introduced above is a plausible
model for how humans actually process images when they view them.  By
considering our response to various changes in the data we view, or in our
internal model of it,  we can make some recommendations for the design of
practical image-viewing robots; likewise, guided by the mathematical rules
followed by our hypothetical robot, we hope to be able to make some hypotheses
about how to carry out a human image inspection discovery program.


% ----------------------------------------------------------------------------

\section{Expectations about Images}

What is our internal model for astronomical images? What prior PDF do we
assign to its parameters?  The answers to these questions are rather different
to the models and priors often used in the image processing literature. There,
for example, gross image properties such as sparseness and positivity have
been used extensively. The prior PDFs associated with these expectations are
so uninformative that the information content of an observed image will always
be high, just because images drawn from these prior distributions do not look
like astronomical images. Features in the image will appear to have 
information $I$ that increases as they depart further from the featureless
background: big galaxies will appear to be more informative  than small  ones,
no matter how interesting the latter are, and so on. 

When viewing astronomical images ourselves we are typically more interested in
the features of images rather than the images themselves, and how interested
we are depends on what we have seen before. A prior PDF for features in
astronomical images is so high dimensional and 
complex that it has not yet been encoded such that a robot can view images
in the manner of a human. However, one way to approach this problem
would be to build one in
the same way that we did for ourselves: we note interesting things in images
based on our experience of a large library of previously viewed images. Humans
are extremely good at interpolating between points in image feature space,
recognising a galaxy as being a galaxy based on their experience of many other
images of galaxies.  What is especially impressive is that this set of images
is typically highly heterogeneous, having varying resolution, depth, dynamic
range and spectral coverage. This means that when we are presented with a new
image, we have to infer its resolution, depth, dynamic range and  spectral
coverage from the data itself, with only fairly broad priors to work from. Had
our training images been displayed {\it homogeneously}, these priors could be
made much narrower, to the point where the properties of the imaging system
could be assumed rather than inferred from the data each time. This would
simplify the robot's task significantly.

Note that in considering such a procedure, we are dividing up our image  model
parameters into those we are interested in (that describe astrophysical
objects) and those we are not (that describe the imaging system). We don't
just want informative images -- we want that information to be about 
astrophysical objects. Image features of non-astrophysical origin are called
artifacts. They  have potentially high surprise value, but are not
interesting. Artifacts can be introduced by either the hardware or software
parts of an imaging system, but either way they provide high information
content unless they are included in the model.  How should we deal with
artifacts to make our robot's task easier? One way would be to include them in
its model, so that it would recognise artifacts and ignore them. The problem
with this approach is one of confusion: if a particular class of artifacts
looks like a new, interesting thing, images that contain examples of those
new, interesting things will be assigned low information content by the robot.
If we are looking for faint, blue, streaks, we had better not
introduce artifacts that are faint, blue and streaky.
Model degeneracy between artifacts and new features is a particularly 
problematic barrier to discovery. Perhaps the best way round it is to  note
that truly informative images are {\it clean}, and assert that the 
fewer artifacts introduced, the better.

{\it A model for features in astronomical images can be built by interpolating
a large sample of library images. This task can be simplified by partitioning
off the volume of parameter space associated with the imaging system itself,
(by preparing a homogeneous set of library images) and by avoiding (where
possible) the need to parametrize uninteresting artifacts (by designing an
imaging system that does not introduce them in the first place).}


% ----------------------------------------------------------------------------

\section{Image Quality}

A viewer with a given model in mind will make discoveries in the images  that
are highly {\it informative}. What do we mean by this? Consider the following
examples of high information content images, ones with high ``surprise value''
that were found to contain:  an unusual faint object was spotted;  an
otherwise typical bright object was seen to actually be two objects; a
familiar-looking object was found to have unusual colour.  In each of these
examples, the feature was somewhat familiar, but some property of the image
itself enabled the surprise of the feature: \it{angular resolution, depth and
colour are all clearly related to information content}. 

The robot's information content, $I(\boldsymbol{d}|H)$, can be seen to depend
on these image properties too, as follows. Recall that in order to compute 
$I(\boldsymbol{d}|H)$, the robot has to predict the image by varying the
parameters of its model, compute the likelihood of each predicted image, and
then integrate over all its model parameters to compute the evidence for the
model. Increasing the signal to noise of an image (by collecting more photons
such that the fractional pixel uncertainty due to Poisson fluctuations is
reduced) has the effect of steepening the likelihood function and so reducing
the volume of parameter space associated with high goodness of fit. Model
features that fitted low signal to noise images badly will give even worse
fits when the signal to noise improves. Increasing the resolution of an image
has a similar effect: where before, a range of model features  differing only
in unresolved detail all gave comparably high likelihoods, with higher
resolution only a fraction of those models will still be good fits, and so the
marginal likelihood will decrease and $I$ will increase. Likewise, adding more
spectral coverage, by including images in more filters, means that a narrower
range of model features will fit the data -- the modeller has to get the
colours right as well as the shapes.  

How can we increase the informativeness of a displayed image? Astronomical
image signal to noise, or depth, is limited by the telescope collecting area,
throughput and the exposure time used; angular resolution is similarly limited
by either the telescope aperture size, or the stability of the atmosphere, or
in some cases the motion of the telescope, and so on. The choice of filters
used determines the spectral coverage of the image set. In principle, an image
viewing robot could take all these factors into account when computing its
predicted images, making a high fidelity model image of the sky, and then
degrading it in order to match the observed image and  compute its likelihood.
In practice, however, such calculations may be too computationally expensive
to perform in the available time. Instead, intermediate images with enhanced
resolution and apparent signal to noise can be produced by inferring model
images assuming uninformative priors. Such inferences incorporate two pieces
of information that the robot has, but may not be able to use efficiently: the
point spread function in the image, and a noise model. Such intermediate image
reconstruction is the subject of much of the image restoration literature
referred to above. The output model images are ``deconvolved,'' and have
higher angular resolution, and smaller (although perhaps correlated) pixel
uncertainties. However, they may also contain new artifacts. Whether the
introduction of these artifacts negates the benefits of the deconvolution
analysis will depend on the discoverable  features in the image. 

Do we human viewers carry out an approximation to this restoration procedure?
Information about the point spread function of the image is available via the
stars present -- but do we use this information and perform some correction --
deconvolution -- when viewing the image? To what extent are we able to correct
for a particular set of filters having been used, so that we can compare with
our model images in some other set? Are we able to reliably tell the
difference between noise and real, but faint, features in an image? Answering
these questions would require a controlled study in feature detection in
images by human viewers, something that is now possible on a large scale via
the online citizen science platform, the Zooniverse. Images of the same
simulated scene both before and after recontruction could be shown to the
assembled crowd, and their detection rate of various features analyzed. It is
important to do this test with simulated data, to be able to measure absolute
detection sensitivity rather than that relative to a set of ``expert''
viewers.

Deconvolution to obtain a lower noise, higher resolution model image is a
procedure is most straightforwardly carried out on a single filter image. The
point spread function varies, in general, with time and wavelength: every
image's PSF is different. This means that any artifacts introduced by the
deconvolution process are also likely to be different between images of
different filters, leading to complex patterns in a color composite image.
However, we note that combining raw images in three filters into a composite
{\it without} deconvolving them to the same resolution will also cause
artifacts: colored features that are not real, but simply a product
of the three filters' images having different PSFs and depths. Combining images into
color composites would seem to produce artifacts generically: determining 
which method of color image production cause the most confusing set of
artifacts will likely depend on the discoverable features present and the
image quality, and is a matter for experiment.

{\it Our ability to improve and extend our model for features in astronomical
images depends on the quality of images available. Beyond the limits imposed
by the image acquisition hardware, we can improve the resolution and signal to
noise of images by inferring intermediate model images, at the price of
introducing new, colored artifacts. However, since colored artifacts are
introduced by forming color composites from input images with different
resolutions and depths anyway, the benefits of deconvolution must be
determined in context by the viewing of test, simulated images.}


% ----------------------------------------------------------------------------

\section{Practical Image Preparation and Viewing}

Suppose we are given three images, labelled red green and blue, with different
depth, and resolution. How should we best present these data to human viewers,
to produce a homogeneous set of  maximally informative images? Based on the
discussion above, we recognise the following useful procedures:

\begin{description}

\item{Resolution matching}. Infer three model images with the same angular
resolution so that,  when they are combined, produce no new color artifacts.
If this can be achieved, the image will appear to have been made with a system
with constant angular resolution. All other images in the set should be
prepared with this resolution as well, to make a homogeneous viewing set.

\item{Deconvolution.} Choose the model image resolution to be as high as
possible, while preventing the introduction of new artifacts. This trade-off
must be understood and quantified. 

\item{Standard stretching.} Display the composite images with a common stretch
and scale such that they appear to have been taken with a color camera of
pre-determined sensitivity.

\end{description}


% Since it is, by definition, dependent on the feature prior, information
% content is clearly subject specific.  The informativeness of an image will
% also be subjective because we each have our own sensor system through which to
% pass the composite image. The color balance, contrast, brightness and dynamic
% range of the image may need to be adjustable by the viewer to account for
% this. However, control of these image properties may be able to be simplified
% to a small number of parameters. These parameters would then need to be used
% consistently on all images in a set for homogeneity to be preserved: it may be
% simplest to fix the presentation for all viewers and live with the resulting
% scatter in discovery sensitivity.

% ----------------------------------------------------------------------------

\section{Testing Discovery Sensitivity in Human Viewers}

Methods, experimental set up. 

Test images. Some specific features to discover.

% Suugestion: A practical experiment would be to view a training set of N
% images, and then measure how surprised you are by M test images. To test
% information content, a different M images could be prepared, and the mean
% surprise level measured and compared to what it was before.  Surprise could be
% measured by questionnaire - did you see the faint blue streak? 

% ----------------------------------------------------------------------------







\end{document}
